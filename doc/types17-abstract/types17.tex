\documentclass[a4paper]{easychair}

\usepackage{bbold}
\usepackage{amssymb}
\usepackage{mathpartir}

\newtheorem{lemma}{Lemma}

\title{Typing with Leftovers: a Mechanisation of ILL}
\titlerunning{Typing with Leftovers}

\author{Guillaume Allais \inst{1}}
\institute{
  Radboud University
  Nijmegen, The Netherlands
  \email{gallais@cs.ru.nl}}
\authorrunning{Allais G.}

\begin{document}
\maketitle

In a linear type system, all the resources available in the context
have to be used exactly once by the term being checked. In traditional
presentations of intuitionistic linear logic this is achieved by
representing the context as a multiset which, in each rule, gets cut
up and distributed among the premises. This is epitomised
by the right rule for tensor (cf. Figure~\ref{rule:tensor}).

However, multisets are an intrinsically extensional notion and
therefore quite arduous to work with in an intensional type
theory. Various strategies can be applied to tackle this issue;
most of them rely on using linked lists to represent contexts
together with either extra inference rules to reorganise the
context or a side condition to rules splitting the context so
that it may be re-arranged on the fly. In the following example,
$≈$ stands for ``bag-equivalence'' of lists.

\begin{figure}[ht]
\begin{mathpar}
\inferrule
 {Γ ⊢ σ \and Δ ⊢ τ
}{Γ, Δ ⊢ σ ⊗ τ
}{⊗_i}

\and \inferrule
 {Γ ⊢ σ \and Δ ⊢ τ \and Γ, Δ ≈ Θ
}{Θ ⊢ σ ⊗ τ
}{⊗_i}
\end{mathpar}
\caption{Introduction rules for $⊗$.
         Left: usual presentation;
         Right: with reordering on the fly\label{rule:tensor}}
\end{figure}

All of these strategies are artefacts of the unfortunate mismatch
between the ideal mathematical objects one wishes to model and
their internal representation in the proof assistant. Short of
having proper quotient types, this will continue to be an issue
when dealing with multisets. The solution we offer tries not to
replicate a set-theoretic approach in intuitionistic type theory
but rather strives to find the type theoretical structures which
can make the problem more tractable. Given the right abstractions,
the proofs follow directly by structural induction.

McBride's recent work~\cite{McBride2016} on combining linear and
dependent types highlights the distinction one can make between
referring to a resource and actually consuming it. In the same spirit,
rather than dispatching the available resources in the appropriate
sub-derivations, we consider that a term is checked in a \emph{given}
context (typically named $γ$) on top of which usage annotations
(typically named $Γ$, $Δ$, etc.) for each of its variables are
super-imposed.

A derivation $Γ ⊢ σ ⊠ Δ$ corresponds to a proof of $σ$ in the underlying
context $γ$ with input (resp. output) usage annotations $Γ$ (resp. $Δ$).
Informally, the resources used to prove $σ$ correspond to a subset of
$γ$: they are precisely the ones which used to be marked \emph{free}
in the input usage annotation and come out marked \emph{stale} in the
\emph{leftovers} $Δ$.

Wherever we used to split the context between sub-derivations, we now
thread the leftovers from one to the next. Writing $f_{σ}$ for a
\emph{fresh} resource of type $σ$ and $s_{σ}$ for a \emph{stale} one,
we can give new introduction rules for the variable with de Bruijn
index zero, tensor and the linear implication, three examples of the
treatment of context annotation, splitting and extension:

\begin{figure}[ht]
\begin{mathpar}
\inferrule
 {
}{Γ ∙ f_{σ} ⊢ σ ⊠ Γ ∙ s_{σ}
}{var_0}

\and \inferrule
 {Γ ⊢ σ ⊠ Δ \and Δ ⊢ τ ⊠ Θ
}{Γ ⊢ σ ⊗ τ ⊠ Θ
}{⊗_i}

\and \inferrule
 {Γ ∙ f_{σ} ⊢ τ ⊠ Δ ∙ s_{σ}
}{Γ ⊢ σ ⊸ τ ⊠ Δ
}{⊸_i}
\end{mathpar}
\caption{Introduction rules for $var_0$ $⊗$ and $⊸$ with leftovers\label{rules:leftovers}}
\end{figure}

This approach is particularly well-suited to use intuitionistic linear
logic as a type system for an untyped $λ$-calculus where well-scopedness
is statically enforced: in the untyped calculus, it \emph{is} the case
that both branches of a pair live in the same scope. In our development,
we use an inductive family in the style of Altenkirch and Reus~\cite{Altenkirch1999}
and opt for a bidirectional presentation~\cite{Pierce:2000:LTI:345099.345100}
to minimise the amount of redundant information that needs to be stored.

Type Inference (resp. Type Checking) is then inferring (resp. checking)
a term's type but \emph{also} annotating the resources it consumed and
returning the \emph{leftovers}. These typing relations can be described
by two mutual definitions; e.g. the definitions in Figure~\ref{rules:leftovers}
would become:

\begin{figure}[ht]
\begin{mathpar}
\inferrule
 {
}{Γ ∙ f_{σ} ⊢ v_0 ∈ σ ⊠ Γ ∙ s_{σ}
}

\and \inferrule
 {Γ ⊢ σ ∋ a ⊠ Δ \and Δ ⊢ τ ∋ b ⊠ Θ
}{Γ ⊢ σ ⊗ τ ∋ (a, b) ⊠ Θ
}

\and \inferrule
 {Γ ∙ f_{σ} ⊢ τ ∋ b ⊠ Δ ∙ s_{σ}
}{Γ ⊢ σ ⊸ τ ∋ λb ⊠ Δ
}
\end{mathpar}
\caption{Type \emph{Inference} rule for $var_0$ and Type \emph{Checking} rules for pairs and lambdas\label{rules:checking}}
\end{figure}

For this mechanisation to be usable, it needs to be well-behaved with
respect to the natural operations on the underlying calculus (renaming
and substitution) but also encompass all of ILL. Our Agda formalisation
(available at \url{https://github.com/gallais/typing-with-leftovers})
states and proves the following results for a system handling types of
the shape:
\[σ, τ, ... ::= α \,|\, 𝟘 \,|\, 𝟙 \,|\, σ \,⊸\, τ \,|\, σ \,⊕\, τ \,|\, σ \,⊗\, τ \,|\, σ \,\&\, τ\]

\begin{lemma}[Framing] The usage annotation of resources left untouched
by a typing derivation can be altered freely. The change is unnoticeable
from the underlying $λ$-term's point of view.
\end{lemma}

\begin{lemma}[Weakening] The input and output contexts of a typing
derivation can be expanded with arbitrarily many new resources. This
corresponds to a weakening on the underlying $λ$-term.
\end{lemma}

\begin{lemma}[Parallel Substitution] Given a term and a typing derivation
corresponding to each one of the fresh resources in its typing derivation's
context, one can build a new typing derivation and a leftover environment.
The corresponding action on the underlying $λ$-term is parallel substitution.
\end{lemma}

\begin{lemma}[Functional Relation] The typing relation is functional: for
given ``inputs'', the outputs are uniquely determined. It is also the case
that the input context is uniquely determined by the output one, the term
and the type.
\end{lemma}

\begin{lemma}[Typechecking] Type checking (resp. Type inference) is decidable.
\end{lemma}

\begin{lemma}[Soundness] Typing derivations give rise to sequent proofs
in ILL.
\end{lemma}

\begin{lemma}[Completeness] From a sequent proofs in ILL, one can build a
pair of an untyped term together with the appropriate typing derivation.
\end{lemma}

\bibliographystyle{plain}
\bibliography{main}

\end{document}
