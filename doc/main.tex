\documentclass[a4paper,UKenglish]{lipics-v2016} 
%This is a template for producing LIPIcs articles. 
%See lipics-manual.pdf for further information.
%for A4 paper format use option "a4paper", for US-letter use option "letterpaper"
%for british hyphenation rules use option "UKenglish", for american hyphenation rules use option "USenglish"
% for section-numbered lemmas etc., use "numberwithinsect"

\renewcommand{\textsc}[1]{{\footnotesize \uppercase{#1}}}
\newcommand{\DefinedType}[1]{\text{\textsc{#1}}}

\newcommand{\𝓣}{$\mathcal{T}$}

\newcommand{\append}{+\!\!\!+}

\newcommand{\fresh}[1]{\texttt{f}_{#1}}
\newcommand{\stale}[1]{\texttt{s}_{#1}}

\newcommand{\Usage}[1]{\DefinedType{Usage}_{#1}}
\newcommand{\Usages}[1]{\DefinedType{Usage}_{#1}}

\newcommand{\Type}{\DefinedType{Type}}
\newcommand{\Nat}{\DefinedType{Nat}}
\newcommand{\Var}{\DefinedType{Var}}
\newcommand{\Inferable}{\DefinedType{Infer}}
\newcommand{\Checkable}{\DefinedType{Check}}
\newcommand{\Pattern}{\DefinedType{Pattern}}
\newcommand{\Context}{\DefinedType{Context}}
\newcommand{\OPE}{\DefinedType{OPE}}
\newcommand{\Env}{\DefinedType{Env}}

\newcommand{\Set}{\mathcal{\textit{Set}}}

\usepackage{agda}
\usepackage{float}
\usepackage{mathpartir, lscape}
\usepackage{todonotes}
\usepackage{microtype}
\usepackage{catchfilebetweentags}
\lstset{
    escapeinside='',
    extendedchars=true,
    inputencoding=utf8,
}


\bibliographystyle{plainurl}% the recommended bibstyle

% Author macros::begin %%%%%%%%%%%%%%%%%%%%%%%%%%%%%%%%%%%%%%%%%%%%%%%%
\title{Typing with Leftovers \protect\\
{\Large A mechanization of Intuitionistic Linear Logic\footnote{This work was partially supported by someone.}}}
\titlerunning{Typing with Leftovers} %optional, in case that the title is too long; the running title should fit into the top page column

%% Please provide for each author the \author and \affil macro,
%% even when authors have the same affiliation, i.e. for each
%% author there needs to be the  \author and \affil macros
\author[1]{Guillaume Allais}
\affil[1]{Nijmegen Quantum Logic Group ─ Radboud University\\
  \texttt{gallais@cs.ru.nl}}
\authorrunning{G. Allais} %mandatory. First: Use abbreviated first/middle names. Second (only in severe cases): Use first author plus 'et. al.'

\Copyright{Guillaume Allais}
%mandatory, please use full first names.
% LIPIcs license is "CC-BY";  http://creativecommons.org/licenses/by/3.0/

\subjclass{Dummy classification -- please refer to \url{http://www.acm.org/about/class/ccs98-html}}
% mandatory: Please choose ACM 1998 classifications from http://www.acm.org/about/class/ccs98-html
% . E.g., cite as "F.1.1 Models of Computation".
\keywords{Dummy keyword -- please provide 1--5 keywords}% mandatory: Please provide 1-5 keywords
% Author macros::end %%%%%%%%%%%%%%%%%%%%%%%%%%%%%%%%%%%%%%%%%%%%%%%%%

%Editor-only macros:: begin (do not touch as author)%%%%%%%%%%%%%%%%%%%%%%%%%%%%%%%%%%
\EventEditors{John Q. Open and Joan R. Acces}
\EventNoEds{2}
\EventLongTitle{42nd Conference on Very Important Topics (CVIT 2016)}
\EventShortTitle{CVIT 2016}
\EventAcronym{CVIT}
\EventYear{2016}
\EventDate{December 24--27, 2016}
\EventLocation{Little Whinging, United Kingdom}
\EventLogo{}
\SeriesVolume{42}
\ArticleNo{23}
% Editor-only macros::end %%%%%%%%%%%%%%%%%%%%%%%%%%%%%%%%%%%%%%%%%%%%%%%

\begin{document}

\maketitle

\begin{abstract}
We start from a simple λ-calculus and introduce a bidirectional 
typing relation corresponding to an Intuitionistic Linear Logic. This 
relation is based on the idea that a linear term consumes some of the
resources available in its context whilst leaving behind leftovers
which could then be fed to another program. 

Concretely, this means that typing derivations have both an input 
and an output context. This leads to a notion of weakening (the extra
resources added to the input context come out unchanged in the output
one), a rather direct proof of stability under substitution, an
analogue of the frame rule of separation logic showing that the 
state of unused resources can be safely ignored, and a proof that
typechecking is decidable.\todo{Equivalence with ILL + Surface language}

The work has been fully formalised in Agda, commented source files 
are provided as additional material available at~\url{https://github.com/gallais/typing-with-leftovers}.
\end{abstract}

\section{Introduction}

The strongly-typed functional programming community has benefited from
a wealth of optimisations made possible precisely because the library
author as well as the compiler are aware of the type of the program they
are working on. These optimisations have ranged from Danvy's type-directed
partial evaluation~\cite{Danvy1999Type} residualising specialised programs
to erasure mechanisms --be they user-guided like Coq's extraction~\cite{letouzey2002new}
which systematically removes all the purely logical proofs put in Prop by
the developer or automated like Brady and Teji{\v{s}}c{\'a}k's erased
values~\cite{brady2003inductive,bradypractical}-- and including the library
defining the State-Thread~\cite{launchbury1994lazy} monad which relies on
higher-rank polymorphism and parametricity to ensure the safety of using an
actual mutable object in a lazy, purely functional setting.

However, in the context of the rising development of dependently-typed
programming languages~\cite{Brady2013idris, norell2009dependently} which,
unlike ghc's Haskell~\cite{weirich2013towards}, incorporate a hierarchy
of universes in order to make certain that the underlying logic is consistent,
some of these techniques are not applicable anymore. Indeed, the use of
large quantification in the definition of the ST-monad crucially relies
on impredicativity. As a consequence, the specification of programs
allowed to update a mutable object in a safe way has to change.
Idris has been extended with experimental support for uniqueness types
inspired by Clean's~\cite{achten1993high} and Rust's ownership types~\cite{manual:rust},
all of which stem from linear logic~\cite{girard1987linear}.

In order to be able to use type theory to formally study the meta-theory
of the programming languages whose type system includes notions of linearity,
we need to have a good representation of such constraints.

\paragraph*{Notations} This whole development has been fully formalised
in Agda. Rather than using Agda's syntax, the results are reformulated
in terms of definitions, lemmas, theorems, and examples. However it is
important to keep in mind the distinction between various kinds of objects.
\texttt{Teletype} is used to denote data constructors, \DefinedType{small
capitals} are characteristic of defined types. A type families' index is
written as a subscript e.g. $\Var{}_n$. 

We use two kinds of inference rules to describe inductive families: double
rules are used to define types whilst simple ones correspond to constructors.
In each case the premises correspond to arguments (usually called parameters
and indices for types) and the conclusion shows the name of the constructor.
A typical example is the inductively defined set of unary natural numbers:
\begin{mathpar}
\type{ }{\Nat{} : \Set{}}
\and \constructor{ }{0 : \Nat{}}
\and \constructor{n : \Nat{}}{1+ n : \Nat{}}
\end{mathpar}

\section{The Calculus of Raw Terms}

The calculus we study in this paper is meant to be a core language,
even though it will be rather easy to write programs in it. As a
consequence all the design choices have been guided by the goal
of facilitating its mechanical treatment in a dependently-typed language.
That is why we use de Bruijn indices to represent variable bindings.
We demonstrate in the code accompanying the paper how a simple scope
checking pass can turn a surface level version of the language using
strings as variable names into this representation.

Following Altenkirch and Reus~\cite{altenkirch1999monadic},
we define the raw terms of our language not as an inductive
type but rather as an inductive \emph{family}~\cite{dybjer1994inductive}.
This technique, sometimes dubbed ``type-level de Bruijn indices'',
makes it possible to keep track, in the index of the family, of the
free variables currently in scope. As is nowadays folklore, instead of
using a set-indexed presentation where a closed terms is indexed by
the empty set $⊥$ and fresh variables are introduced by wrapping
the index in a \texttt{Maybe} type constructor\footnote{The value
\texttt{nothing} represents the fresh variable whilst the constructor
\texttt{just} lifts the other ones in the new scope.}, we index
our terms by a natural number instead. The
\Var{} type family\footnote{It is also known as \texttt{Fin} (for
``finite set'') in the dependently typed programming community.}
defined below represents the de Bruijn indices~\cite{debruijn1972lambda}
corresponding to the $n-1$ free variables present in a scope $n$.

\begin{mathpar}
\type
 {n : \Nat{}
}{\Var{}_n : \Set{}
}

\and \constructor
 {
}{\texttt{zero} : \Var{}_{1 + n}
}

\and \constructor
 {k : \Var{}_n
}{\texttt{suc} (k) : \Var{}_{1 + n}
}
\end{mathpar}

We present the calculus in a bidirectional fashion~\cite{pierce2000local}.
This definition style scales well to more complex type theories where full
type-inference is not tractable anymore whilst only requiring the programmer
to add top-level type annotations. The term constructors of the calculus are
split in two different syntactic categories corresponding to constructors
of canonical values on one hand and eliminators on the other. These
categories characterise the flow of information during typechecking: given a
context assigning a type to each free variable, canonical values (which we call
\Checkable{}) can be \emph{check}ed against a type whilst we may \emph{infer}
the type of computations (which we call \Inferable{}).

\todo{Explain what these constructors do?}
\begin{figure}[H]\centering
\begin{tabular}{lcl}
⟨$\Inferable{}_n$⟩ & ::= & \var{} ⟨$\Var{}_n$⟩ \\
                   &  |  & \app{⟨$\Inferable{}_n$⟩}{⟨$\Checkable{}_n$⟩} \\
                   &  |  & \cas{⟨$\Inferable{}_n$⟩}{⟨\Type{})}{⟨$\Checkable{}_{1 + n}$⟩}{⟨$\Checkable{}_{1 + n}$⟩} \\
                   &  |  & \prl{⟨$\Inferable{}_n$⟩} ~|~ \prr{⟨$\Inferable{}_n$⟩} \\
                   &  |  & \cut{⟨$\Checkable{}_n$⟩}{⟨\Type{}⟩} \\ \\

⟨$\Checkable{}_n$⟩ & ::= & \lam{⟨$\Checkable{}_{1 + n}$⟩} \\
                   &  |  & \letin{⟨$\Pattern{}_m$⟩}{⟨$\Inferable{}_n$⟩}{⟨$\Checkable{}_{m + n}$⟩} \\
                   &  |  & \inl{⟨$\Checkable{}_n$⟩} \\
                   &  |  & \inr{⟨$\Checkable{}_n$⟩} \\
                   &  |  & \prd{⟨$\Checkable{}_n$⟩}{⟨$\Checkable{}_n$⟩} \\
                   &  |  & \neu{⟨$\Inferable{}_n$⟩} \\
\end{tabular}
\caption{Grammar of the Language of Raw Terms}
\end{figure}

Two additional rules (\texttt{neu} and \texttt{cut} respectively)
allow the embedding of \Inferable{} into \Checkable{} and vice-versa. They
make it possible to form redexes by embedding canonical values into
computations and then applying eliminators to them. In terms of
typechecking, they correspond to a change of direction between
inferring and checking. The constructor \texttt{cut} takes an
extra \Type{} argument in order to guarantee the success of
type-inference for \Inferable{} terms.

A notable specificity of this language is the ability to use nested
patterns in a let binder rather than having to resort to cascading
lets. This is achieved thanks to a rather simple piece of kit: the
\Pattern{} type family. A value of type $\Pattern{}_n$ represents a
pattern binding $n$ variables. Because variables are represented as
de Bruijn indices, the base pattern does not need to be associated
with a name, it simply is a constructor \texttt{v} binding exactly
one variable. The comma pattern constructor takes two nested patterns
respectively binding $m$ and $n$ variables and uses them to deeply
match a pair thus binding $(m + n)$ variables.

\begin{mathpar}
\type
 {n : \Nat{}
}{\Pattern{}_n : \Set{}
}

\and \constructor
 {
}{\texttt{v} : \Pattern{}_1
}

\and \constructor
 {p : \Pattern{}_m \and q : \Pattern{}_n
}{p \texttt{,} q : \Pattern{}_{m + n}
}
\end{mathpar}

The grammar of raw terms only guarantees that all expressions are
well-scoped by construction. It does not impose any other constraint,
which means that a user may write valid programs but also invalid
ones as the following examples demonstrate:

\begin{example}\label{example:swap}
\texttt{swap} is a closed, well-typed linear term taking a pair as
an input and swapping its components. It corresponds to the mathematical
function $(x, y) \mapsto (y, x)$.
\begin{lstlisting}
  swap = lam (let (v , v) ∷= var zero
              in prd (neu (var (suc zero))) (neu (var zero)))
\end{lstlisting}
\end{example}

\begin{example}\label{example:illTyped}
\texttt{illTyped} is a closed linear term. However it is manifestly
ill-typed: the let-binding it uses tries to break down a function as
if it were a pair.
\begin{lstlisting}
  illTyped = let (v , v) ∷= cut (lam (neu (var zero))) (a ⊸ a)
             in prd (neu (var zero)) (neu (var (suc zero)))
\end{lstlisting}
\end{example}

\begin{example}\label{example:diagonal}
Finally, \texttt{diagonal} is a term typable in the simply-typed
lambda calculus but it is not linear: it duplicates its input just
like $x \mapsto (x, x)$ does.
\begin{lstlisting}
  diagonal = lam (prd (neu (var zero)) (neu (var zero)))
\end{lstlisting}
\end{example}

\section{Linear Typing Rules}

These observations demonstrate that we need to define a typing
relation describing the rules terms need to abide by in order
to qualify as well-typed linear programs. We start by defining
the types our programs may have using the grammar in Figure~\ref{fig:types}.
Apart from the usual linear type formers, we have a constructor
\Base{} which makes it possible to have countably many different
base types.

\begin{figure}[ht]\centering
\begin{tabular}{lcl}
⟨$\Type{}$⟩ & ::= & \Base{⟨$\mathbb{N}$⟩}
              ~|~ \Lolli{⟨$\Type{}$⟩}{⟨$\Type{}$⟩}
              ~|~ \Tensor{⟨$\Type{}$⟩}{⟨$\Type{}$⟩} \\
            &  |  & \Sum{⟨$\Type{}$⟩}{⟨$\Type{}$⟩}
              ~|~ \With{⟨$\Type{}$⟩}{⟨$\Type{}$⟩}
\end{tabular}
\caption{Grammar of $\Type{}$}\label{fig:types}
\end{figure}

A linear type system is characterised by the
fact that all the resources available in a context have to be
used exactly once by the term being checked. In traditional
presentations of linear logic this is achieved by representing
the context as a multiset and, in each rule, cutting it up and
distributing its parts among the premises. This is epitomised
by the introduction rule for tensor (cf. Figure~\ref{rule:tensor}).

However, multisets are an intrinsically extensional notion and
therefore quite arduous to work with in an intensional type
theory. Various strategies can be applied to tackle this issue;
most of them rely on using linked lists to represent contexts
together with either extra inference rules to reorganise the
context or a side condition to rules splitting the context so
that it may be re-arranged on the fly. In the following example
$\_≈\_$ stands for ``bag-equivalence'' of lists.

\begin{figure}[ht]
\begin{mathpar}
\inferrule
 {Γ ⊢ σ \and Δ ⊢ τ
}{Γ, Δ ⊢ σ ⊗ τ
}{⊗_i}

\and \inferrule
 {Γ ⊢ σ \and Δ ⊢ τ \and Γ, Δ ≈ Θ
}{Θ ⊢ σ ⊗ τ
}{⊗_i}
\end{mathpar}
\caption{Introduction rules for tensor (left: usual presentation, right: with reordering on the fly)\label{rule:tensor}}
\end{figure}

All of these strategies are artefacts of the unfortunate mismatch
between the ideal mathematical objects one wishes to model and
their internal representation in the proof assistant. Short of
having proper quotient types, this will continue to be an issue
when dealing with multisets. The solution described in the rest
of this paper tries not to replicate a set-theoretic approach in
intuitionistic type theory but rather strives to find the type
theoretical structures which can make the problem more tractable.
Indeed, given the right abstractions most proofs become simple
structural inductions.

\subsection{Usage Annotations}

McBride's recent work~\cite{mcbride2016got} on combining linear and
dependent types highlights the distinction one can make between
referring to a resource and actually consuming it. In the same spirit,
rather than dispatching the available resources in the appropriate
subderivations, we consider that a term is checked in a \emph{given}
context on top of which usage annotations are super-imposed. These
usage annotations indicate whether resources have been consumed already
or are still available. Type-inference (resp. Type-checking) is then
inferring (resp. checking) a term's type but \emph{also} annotating
the resources consumed by the term in question and returning the
\emph{leftovers} which gave their name to this paper.

\begin{definition}
\label{definition:context}
A \Context{} is a list of \Type{}s indexed by its length. It can
be formally described by the following inference rules:
\begin{mathpar}
\type
 {n : \Nat{}
}{\Context{}_n : \Set{}
}

\and \constructor
 {
}{[] : \Context{}_0
}

\and \constructor
 {γ : \Context{}_n \and σ : \Type{}
}{γ ∙ σ : \Context{}_{1 + n}
}
\end{mathpar}
\end{definition}


\begin{definition}
\label{definition:usage}
A $\Usage{}$ is a predicate on a type $σ$ describing whether the
resource associated to it is available or not. We name the
constructors describing these two states $\fresh{}$ (for ``fresh'')
and $\stale{}$ (for ``stale'') respectively.
\todo{Overloading}
 These are naturally
lifted to contexts in a pointwise manner and we reuse the $\Usage{}$
name and the $\fresh{}$ and $\stale{}$ constructors.

\begin{mathpar}
\type
 {σ : \Type{}
}{\Usage{σ} : \Set{}
}
\and\constructor
 {
}{\fresh{σ} : \Usage{σ}
}
\and\constructor
 {
}{\stale{σ} : \Usage{σ}
}
\end{mathpar}
\begin{mathpar}
\type
 { γ : \Context{}_n
}{\Usages{γ} : \Set{}
}
\and\constructor
 {
}{[] : \Usages{[]}
}
\and\constructor
 {Γ : \Usages{γ ∙ σ} \and S : \Usage{σ}
}{Γ ∙ S : \Usages{γ ∙ σ}
}
\end{mathpar}
\end{definition}

\subsection{Typing as Consumption Annotation}

A Typing relation seen as a consumption annotation process describes
what it means, given a context an its usage annotation, to ascribe a
type to a term whilst crafting another usage annotation containing all
the leftover resources. Formally:

\begin{definition}
\label{definition:typing}
A ``Typing Relation'' for $T$ a \Nat{}-indexed inductive family is
an indexed relation $\text{\𝓣{}}_n$ such that:
\begin{mathpar}
\type
 {n : \Nat{} \and γ : \Context{}_n \and Γ, Δ : \Usages{γ} \and t : T_n \and σ : \Type{}
}{\text{\𝓣{}}_n(Γ, t, σ, Δ) : \Set{}
}
\end{mathpar}
\end{definition}

This definition clarifies the notion but also leads to more generic
statements later on: weakening, substitution, framing can all be
expressed as properties a Typing Relation might have.

\todo{explain}
\begin{figure}[H]
\begin{mathpar}
\type[\Var{}]
 {{\begin{array}{l}
    γ : \Context{n} \\
    Γ : \Usages{γ} \\ \\
  \end{array}}
  \and k : \Var{n}
  \and {\begin{array}{l}
    σ : \Type{}    \\
    Δ : \Usages{γ} \\ \\
  \end{array}}
}{Γ ⊢_v k ∈ σ \andalso{} Δ : \Set{}
}
\and \type[\Inferable{}]
 {{\begin{array}{l}
    γ : \Context{n} \\
    Γ : \Usages{γ} \\ \\
  \end{array}}
  \and t : \Inferable{n}
  \and {\begin{array}{l}
    σ : \Type{}    \\
    Δ : \Usages{γ} \\ \\
  \end{array}}
}{Γ ⊢ t ∈ σ \andalso{} Δ : \Set{}
}
\and \type[\Checkable{}]
 {{\begin{array}{l}
    γ : \Context{n} \\
    Γ : \Usages{γ} \\
    σ : \Type{} \\ \\
   \end{array}}
  \and t : \Checkable{n}
  \and Δ : \Usages{γ}
}{Γ ⊢ σ  ∋ t \andalso{} Δ : \Set{}
}
\and \type[\Pattern{}]
{ σ : \Type{}
  \and p : \Pattern{n}
  \and γ : \Context{n}
}{σ ∋ p \leadsto{} γ : \Set{}
}
\end{mathpar}
\caption{Typing relations for \Var{}, \Inferable{}, \Checkable{} and \Pattern{}}
\end{figure}


\subsubsection{Typing de Bruijn indices}

The simplest instance of a Typing Relation is the one for de Bruijn
indices: given an index $k$ and a usage annotation, it successfully
associates a type $σ$ to that index if and only if the $k$th resource
in context is of type $σ$ and fresh (i.e. its $\Usage{}$ is $\fresh{σ}$).
In the resulting leftovers, this resource will have turned stale ($\stale{σ}$):
\begin{definition}
\label{typing:deBruijn}
The typing relation is presented in a sequent-style: Γ ⊢ $k$ ∈ σ ⊠ Δ
means that starting from the usage annotation Γ, the de Bruijn index
$k$ is ascribed type σ with leftovers Δ. It is defined inductively by
two constructors (cf. Figure~\ref{figure:deBruijn}).
\end{definition}

\begin{remark}The careful reader will have noticed that there is precisely
one typing rule for each \Var{} constructor. It is not a coincidence. And
if these typing rules are not named it's because in Agda, they can simply
be given the same name as their \Var{} counterpart. The same will be true
for \Inferable{}, \Checkable{} and \Pattern{} which means that writing
down a typable program could be seen as either writing a raw term or the
typing derivation associated to it depending on the author's intent.
\end{remark}

\begin{example}
The de Bruijn index 1 has type τ in the context (γ ∙ σ ∙ τ) with
usage annotation ($Γ ∙ \fresh{τ} ∙ \fresh{σ}$):
\begin{mathpar}
\inferrule
 {\inferrule
   {
  }{Γ ∙ \fresh{τ} ⊢ \varzero{} ∈ τ ⊠ Γ ∙ \stale{τ}
  }
}{Γ ∙ \fresh{τ} ∙ \fresh{σ} ⊢ \varsucc{\varzero} ∈ τ ⊠ Γ ∙ \stale{τ} ∙ \fresh{σ}
}
\end{mathpar}
Or, as it would be written in Agda, taking advantage of the fact that
the language constructs and the typing rules about them have been given
the same names:
\begin{lstlisting}
  one : 'Γ' '∙' f 'τ' '∙' f 'σ' ⊢ s z '∈' 'τ' '⊠' 'Γ' '∙' s 'τ' '∙' f 'σ'
  one = suc zero
\end{lstlisting}
\end{example}

\subsubsection{Typing Terms}

The key idea appearing in all the typing rules for compound
expressions is to use the input $\Usages{}$ to type one of the
sub-expressions, collect the leftovers from that typing
derivation and use them as the new input $\Usages{}$ when typing
the next sub-expression.

Another common pattern can be seen across all the rules involving
binders, be they λ-abstractions, let-bindings or branches of a
case. Typechecking the body of a binder involves extending the
input $\Usages{}$ with fresh variables and observing that they have
become stale in the output one. This guarantees that these bound
variables cannot escape their scope as well as that they have indeed
been used. Relaxing the staleness restriction would lead to an affine
type system which would be interesting in its own right.

\begin{definition}The Typing Relation for \Inferable{} is typeset
in a fashion similar to the one for \Var{}. Indeed, in both cases
the type is inferred. $Γ ⊢ t ∈ σ ⊠ Δ$ means that given Γ a
$\Usages{γ}$, and $t$ an \Inferable{}, the type σ is inferred
together with leftovers Δ, another $\Usages{γ}$. The rules are
listed in Figure~\ref{figure:infer}.

For \Checkable{}, the type σ comes first: $Γ ⊢ σ ∋ t ⊠ Δ$ means
that given Γ a $\Usages{γ}$, a type σ, the \Checkable{} $t$ can
be checked to have type σ with leftovers Δ. The rules can be found
in Figure~\ref{figure:check}.

Finally, \Pattern{}s are checked against a type and a context of
newly bound variables is generated. If the variable pattern always
succeeds, the pair constructor pattern on the other hand obviously
only succeeds if the type it attempts to split is a tensor type.
The context of newly-bound variables is then the collection of the
contexts associated to the nested patterns. The rules are given in
Figure~\ref{figure:pattern}.
\end{definition}

\begin{example}
Given these rules, it is easy to see that the identity function
can be checked at type (σ ⊸ σ) in an empty context:
\begin{mathpar}
\inferrule
 {\inferrule
   {\inferrule
     {\inferrule
       {
      }{[] ∙ \fresh{σ} ⊢ \varzero ∈ σ ⊠ [] ∙ \stale{σ}
      }
    }{[] ∙ \fresh{σ} ⊢ \var{\varzero} ∈ σ ⊠ [] ∙ \stale{σ}
    }
  }{[] ∙ \fresh{σ} ⊢ σ ∋ \neu{(\var{\varzero})} ⊠ [] ∙ \stale{σ}
  }
}{[] ⊢ σ ⊸ σ ∋ \lam{(\neu{(\var{\varzero})})} ⊠ []
}
\end{mathpar}
Or, as it would be written in Agda where the typing rules were
given the same name as their term constructor counterparts:
\begin{lstlisting}
  identity : [] '⊢' 'σ' '⊸' 'σ' '∋' lam (neu (var z)) '⊠' []
  identity = lam (neu (var z))
\end{lstlisting}
\end{example}

\begin{example}\label{example:swapTyped}
It is also possible to revisit Example \ref{example:swap} to prove
that it can be checked against type (σ ⊗ τ) ⊸ (τ ⊗ σ) in an empty
context. This gives the lengthy derivation included in the appendix
or the following one in Agda which quite a lot more readable:

\begin{lstlisting}
  swapTyped : [] '⊢' ('σ' '⊗' 'τ') '⊸' ('τ' '⊗' 'σ') '∋' swap '⊠' []
  swapTyped = lam (let (v , v) ∷= var z
                   in prd (neu (var (s z))) (neu (var z))
\end{lstlisting}
\end{example}

%%%%%%%%%%%%%
%% FRAMING %%
%%%%%%%%%%%%%

\section{Framing}

The most basic property one can prove about this typing system is
the fact that the state of the resources which are not used by a
lambda term is irrelevant. We call this property the Framing
Property because of the obvious analogy with the frame rule in
separation logic. This can be reformulated as the fact that as
long as two pairs of an input and an output usages exhibit the
same consumption pattern then if a derivation uses one of these,
it can use the other one instead. Formally (postponing the
definition of $Γ - Δ ≡ Θ - ξ$):

\begin{definition}A Typing Relation \𝓣{} for a \Nat{}-indexed
family $T$ has the Framing Property if for all $k$ a \Nat{},
γ a $\Context{}_k$, Γ, Δ, Θ, ξ four $\Usages{γ}$, $t$ an element
of $T_k$ and σ a Type, if $Γ - Δ ≡ Θ - ξ$ and \𝓣{}$(Γ, t, σ, Δ)$
then \𝓣{}$(Θ, t, σ, ξ)$ also holds.
\end{definition}


\begin{definition}
\label{definition:differences}
The ``consumption equivalence'' characterises the pairs of an input and
an output $\Usages{}$ which have the same consumption pattern. The
usages annotations for the empty context are trivially related.
If the context is not empty, then there are two cases: if the
resource is left untouched on one side, then so should it on the other
side but the two annotations may be different (here denoted $A$ and $B$
respectively). On the other hand, if the resource has been consumed
on one side then it has to be on the other side too.
\begin{mathpar}
\type
 {Γ, Δ, Θ, ξ : \Usages{γ}
}{Γ ─ Δ ≡ Θ ─ ξ : \Set{}
}
\and \constructor
 {
}{[] - [] ≡ [] - []
}{
}
\and \constructor
 {Γ - Δ ≡ Θ - ξ
}{(Γ ∙ A) - (Δ ∙ A) ≡ (Θ ∙ B) - (ξ ∙ B)
}{
}
\and \constructor
 {Γ - Δ ≡ Θ - ξ
}{(Γ ∙ \texttt{fresh}_σ) - (Δ ∙ \texttt{stale}_σ) ≡ (Θ ∙ \texttt{fresh}_σ) - (ξ ∙ \texttt{stale}_σ)
}{
}
\end{mathpar}
\end{definition}

\begin{definition}The ``consumption partial order'' $Γ ⊆ Δ$ is defined as
$Γ - Δ ≡ Γ - Δ$.
\end{definition}

\begin{lemma}
\begin{enumerate}
  \item The consumption equivalence is a partial equivalence
  \item The consumption partial order is a partial order
  \item If there is a Usages ``in between'' two others according to the consumption
        partial order, then any pair of usages equal to these two can be split in a
        manner compatible with this middle element. Formally: Given Γ, Δ, Θ, ξ
        and χ such that $Γ - Δ ≡ Θ - ξ$, $Γ ⊆ χ$ and $χ ⊆ Δ$,
        one can find ζ such that: $Γ - χ ≡ Θ - ζ$ and $χ - Δ ≡ ζ - ξ$.
\end{enumerate}
\end{lemma}

\begin{lemma}[Consumption]The Typing Relations for \Var{}, \Inferable{}
and \Checkable{} all imply that if a typing derivation exists with input
usages annotation Γ and output usages annotation Δ then $Γ ⊆ Δ$.
\end{lemma}

\begin{theorem}
\label{theorem:framing}
The Typing Relations for \Var{} has the Framing Property.
So do the ones for \Inferable{} and \Checkable{}.
\end{theorem}
\begin{proof}
The proofs are by structural induction on the typing derivations.
They rely on the previous lemmas to generate the inclusion evidence
and use it to split up the witness of consumption equivalence and
distribute it appropriately in the induction hypotheses.
\end{proof}

%%%%%%%%%%%%%%%
%% WEAKENING %%
%%%%%%%%%%%%%%%

\section{Weakening}

It is perhaps surprising to find a notion of weakening for a linear
calculus: the whole point of linearity is precisely to ensure that
all the resources are used. However when opting for a system based
on consumption annotations it becomes necessary, in order to define
substitution for instance, to be able to extend the underlying
context a term is defined with respect to. Linearity is guaranteed
by ensuring that the inserted variables are left untouched by the
term.

Weakening arises from a notion of inclusion. The appropriate type
theoretical structure to describe these inclusions is well-known
and called an Order Preserving Embeddding~\cite{chapman2009thesis,altenkirch1995categorical}.
Unlike a simple function witnessing the inclusion of its domain
into its codomain, the restriction brought by order preserving
embeddings guarantees that contraction is simply not possible which
is crucial in a linear setting. In this development, three variations
on OPEs are actually needed: one for \Nat{}s to describe the embedding
of a scope into a larger one, one for \Context{} describing what types
the extra variables in scope are assigned and finally one for their
$\Usages{}$ mentioning whether these variables are fresh or stale.

\begin{definition}
An \textbf{Order Preserving Embedding} (OPE) is an inductive family. Its
constructors dubbed ``moves'' describe a strategy to realise the promise
of an injective embedding which respects the order induced by the de Bruijn
indices. We start with an example in Figure~\ref{figure:exampleOpe} before
giving, in Figure~\ref{figure:ope}, the formal definition of OPEs for \Nat{},
\Context{} and $\Usages{}$.
\end{definition}

In the following example, we prove that the source context $\gamma ∙ \tau$
can be safely embedded into the target one $\gamma ∙ \sigma ∙ \tau ∙ \nu$.
To do so, we use all three of the moves the inductive definition of OPEs
offers: $\texttt{insert}_{\alpha}$ which introduces a new variable of type
$\alpha$, $\texttt{copy}$ which embeds the source context's top variable,
and $\texttt{done}$ which simply copies the source context. To clarify the
effect of each move on the source context, we write it in \textit{cons}-style
rather than the \textit{snoc}-style used in the rest of the paper and insert
extra space to align the source elements and their respective embeddings in
the target.

\begin{figure}[ht]\centering
\begin{tabular}{l|ccccccc}
\textit{OPE} & $\texttt{insert}_{\nu}$ &
             & \texttt{copy} &
             & $\texttt{insert}_{\sigma}$ &
             & \texttt{done}\\
\textit{source} & & 
                & $\tau$ & , 
                & &
                & $\gamma$ \\
\textit{target} & $\nu$ & ,
                & $\tau$ & ,
                & $\sigma$ & ,
                & $\gamma$ \\
\end{tabular}
\caption{Example of an Order Preserving Embedding proving:
         $\gamma ∙ \tau
          \subseteq
          \gamma ∙ \sigma ∙ \tau ∙ \nu$\label{figure:exampleOpe}}
\end{figure}

\todo{transition}

The first column lists the names of the constructors associated to each
one of the moves. The second column describes the OPE for \Nat{}s
and it follows closely the traditional definition of OPEs. The two
remaining columns are a bit different: they respectively define the
OPE for \Context{}s and the one for $\Usages{}$. However they do not
mention the source and target sets in their indices; they are in
fact generic enough to be applied to any context / usages of the
right size. If a value of $k ≤ l$ is seen as a \emph{diff} between
$k$ and $l$ then the OPEs on contexts and usages annotations only
specify what to do for the variables introduced by the diff (i.e.
the variables corresponding to an \texttt{insert} constructor).
These diffs can then be applied to concrete contexts and usages
respectively to transform them.

The first row defines the move \texttt{done} for each one of the
OPEs. It is the strategy corresponding to the trivial embedding of
a set into itself by the identity function. It serves as a base case.

The second row corresponds to the \texttt{copy} move which extends
an existing embedding by copying the current $0$th variable from
source to target. The corresponding cases for \Context{}s and
$\Usages{}$ are purely structural: no additional content is required
to be able to perform a \texttt{copy} move.

Last but not least, the third row describes the move \texttt{insert}
which introduces an extra variable in the target set. This is the
move used to extend an existing context, i.e. to weaken it. In this
case, it is paramount that the OPE for \Context{}s should take a
type σ as an extra argument (it will be the type of the newly introduced
variable) whilst the OPE for $\Usages{}$ takes a $\Usage{σ}$ (it will
be the usage associated to that newly introduced variable of type σ).

\begin{figure}[ht]\centering
\begin{tabular}{l|c|c|c}
& \type
 {k, l : \Nat{}
}{k ≤ l : \Set{}
}
& \type
 {o : k ≤ l 
}{\OPE{}(o) : \Set{}
}
& \type
 {o : k ≤ l \and O : \OPE{}(o)
}{\OPE{}(O) : \Set{}
} \\ & & \\
\texttt{done}
& \constructor
 {
}{k ≤ k
}
& \constructor
 {
}{\OPE{}(\texttt{done})
}
& \constructor
 {
}{\OPE{}(\texttt{done})
}\\ & & \\
\texttt{copy}
& \constructor
 {k ≤ l
}{1+k ≤ 1+l
}
& \constructor
 {o : k ≤ l \and \OPE{}(o)
}{\OPE{}(\texttt{copy}(o))
}
& \constructor
 {{\begin{array}{l}o : k ≤ l \\ O : \OPE{}(o)\end{array}} \and \OPE{}(O)
}{\OPE{}(\texttt{copy}(O))
}\\ & & \\
\texttt{insert}
& \constructor
 {k ≤ l
}{k ≤ 1+l
}
& \constructor
 {o : k ≤ l \and \OPE{}(o) \and \Type{}
}{\OPE{}(\texttt{insert}(o))
}
& \constructor
 {{\begin{array}{l}o : k ≤ l \\ O : \OPE{}(o) \\ σ : \Type{}\end{array}} \and \OPE{}(O) \and S : \Usage{σ}
}{\OPE{}(\texttt{insert}(O, σ))
}
\end{tabular}
\caption{Order Preserving Embeddings for \Nat{}, \Context{} and $\Usages{}$\label{figure:ope}}
\end{figure}


\begin{example}
\label{example:ope}
We may define three embeddings corresponding to the introduction of a
fresh variable for scopes, contexts and usages respectively. The
body of these three declarations looks the same because we overload
the constructors' names but they build different objects. As can be
seen in the types, the latter ones depend on the former ones. This
type of embedding is very much grounded in reality: it is precisely
what pushing a substitution under a lambda abstraction calls for.
\begin{lstlisting}
  scopeWithFV : n '≤' suc n
  scopeWithFV = insert done

  contextWithFV : Type '→' Context.OPE scopeWithFV
  contextWithFV 'σ' = insert done 'σ'

  usagesWithFV : ('σ' : Type) '→' Usage σ '→' Usages.OPE (contextWithFV 'σ')
  usagesWithFV 'σ' S = insert done S
\end{lstlisting}
\end{example}

We leave out the definitions of the two \texttt{patch} functions applying
the diff described by an OPE to respectively a context or a usages
annotation. They are structural in the OPE. The interested reader will find
them in the formal development in Agda. We also leave out the definition of
weakening for raw terms. It is given by a simple structural induction on the
terms themselves.

\begin{definition}A Typing Relation \𝓣{} for a \Nat{}-indexed family $T$
such that we have a function $\texttt{weak}_T$ transporting proofs that
$k ≤ l$ to functions $T_k → T_l$ is said to have the Weakening Property
if for all $k, l$ in \Nat{}, $o$ a proof that $k ≤ l$, $O$ a proof that
$\OPE{}(o)$ and $𝓞$ a proof that $\OPE{}(O)$ then for all γ a $\Context{}_k$,
Γ and Δ two $\Usages{γ}$, $t$ an element of $T_k$ and σ a \Type{}, if
\𝓣{}$(Γ, t, σ, Δ)$ holds true then we also have
\𝓣{}$(\texttt{patch}(𝓞, Γ), \texttt{weak}_T(o, t), σ, \texttt{patch}(𝓞, Δ))$.
\end{definition}

\begin{theorem}The Typing Relation for \Var{} has the Weakening Property.
So do the Typing Relations for \Inferable{} and \Checkable{}.
\end{theorem}
\begin{proof}
The proof for \Var{} is by induction on the typing derivation. The
statements for \Inferable{} and \Checkable{} are proved by mutual
structural inductions on the respective typing derivations. Using the
\texttt{copy} constructor of OPEs is crucial to be able to go under
binders.
\end{proof}



%%%%%%%%%%%%%%%%%%
%% SUBSTITUTION %%
%%%%%%%%%%%%%%%%%%

\section{Substituting}

Stability of the typing relation under substitution guarantees
that the evaluation of programs will yield results which have
the same type as well as preserve the linearity constraints.
The notion of leftovers naturally extends to substitutions: the
terms meant to be substituted for the variables in context which
are not used by a term will not be used when pushing the substitution
onto this term. They will therefore have to be returned as leftovers.

Because of this rather unusual behaviour for substitution, picking
the right type-theoretical representation for the environment
carrying the values to be substituted in is a bit subtle. Indeed,
relying on the usual combination of weakening and crafting a fresh
variable when going under a binder becomes problematic. The leftovers
returned by the induction hypothesis would then live in an extended
context and quite a lot of effort would be needed to downcast them
back to the smaller context they started in. The solution is to have
an explicit constructor for ``going under a binder'' which can be
simply peeled off on the way out of a binder. 

\begin{definition}The environment \Env{} used to define substitution
for raw terms is indexed by two \Nat{}s $k$ and $l$ where $k$ is the
source's scope and $l$ is the target's scope. There are three constructors:
one for the empty environment ([]), one for going under a binder (\texttt{∙v})
and one to extend an environment with an $\Inferable{}_l$.
\begin{mathpar}
\type
 {
}{[] : \Env{}(0, l)
}
\and \constructor
 {ρ : \Env{}(k, l)
}{ρ \texttt{∙v} : \Env{}(\texttt{suc}(k), \texttt{suc}(l)) 
}
\and \constructor
 {ρ : \Env{}(k, l) \and t : \Inferable{}_l
}{ρ ∙ t : \Env{} (\texttt{suc}(k), l)
}
\end{mathpar}
\end{definition}

Environment are carrying \Inferable{} elements because, being in the
same syntactical class as \Var{}s, they can be substituted for them
without any issue.

\begin{lemma}Raw terms are stable under substitutions: for all $k$ and
$l$, given $t$ a term $\Inferable{}_k$ (resp. $\Checkable{}_k$) and $ρ$
an environment $\Env{}(k,l)$, we can apply the substitution $ρ$ to $t$
and obtain an $\Inferable{}_l$ (resp. $\Checkable{}_l$).
\end{lemma}
\begin{proof}By mutual induction on the raw terms, using the \texttt{∙v}
\Env{} constructor when going under a binder. The need for weakening or
crafting fresh variables has not disappeared, it has been transferred to
the auxiliary function looking up a value in $ρ$ given a $\Var{}_k$.
\end{proof}

\begin{definition}The environments used when proving that Typing
Relations are stable under substitution follow closely the ones
for raw terms. $\Env{}(Θ₁, ρ, Θ₂, Γ)$ is a typing relation with
input usages $Θ₁$ and output $Θ₂$ for the raw substitution $ρ$
targeting the \texttt{fresh} variables in $Γ$. Unsurprisingly,
the typing for the empty environment has the same input and output
usages annotation. Formally:
\begin{mathpar}
\type
 {{\begin{array}{l}θ : \Context{}_l \\ Θ₁, Θ₂ : \Usages{θ}\end{array}} \and ρ : \Env{}(k,l)
  \and {\begin{array}{l}γ : Context{}_k \\ Γ : \Usages{γ}\end{array}}
}{\Env{}(Θ₁, ρ, Θ₂, Γ) : \Set{}
}
\and \constructor
 {
}{\Env{}(Θ₁, [], Θ₁, [])
}
\end{mathpar}
For \texttt{fresh} variables in Γ, there are two cases depending whether
they have been introduced by going under a binder or not. If it is
not the case then the typing environment carries around a typing
derivation for the term $t$ meant to be substituted for this variable.
Otherwise, it does not carry anything extra but tracks in its input /
output usages annotation the fact the variable has been consumed.
\begin{mathpar}
\constructor
 {Θ₁ ⊢ t ∈ σ ⊠ Θ₂ \and \Env{}(Θ₂, ρ, Θ₃, Γ)
}{\Env{}(Θ₁, ρ ∙ t, Θ₃, Γ ∙ \texttt{fresh}_σ)
}
\and \constructor
 {\Env{}(Θ₁, ρ, Θ₂, Γ)
}{\Env{}(Θ₁ ∙ \texttt{fresh}_σ, ρ \texttt{∙v}, Θ₂ ∙ \texttt{stale}_σ, Γ ∙ \texttt{fresh}_σ)
}
\end{mathpar}
For \texttt{stale} variables, there are two cases too. They are however
a bit more similar: none of them carry around an extra typing derivation.
The main difference is in the shape of the input and output context: in
the case for the ``going under a binder'' constructor, they are clearly
enriched with an extra (now consumed) variable whereas it is not the case
for the normal environment extension.
\begin{mathpar}
\constructor
 {\Env{}(Θ₁, ρ, Θ₂, Γ)
}{\Env{}(Θ₁, ρ ∙ t, Θ₂, Γ ∙ \texttt{stale}_σ)
}
\and \constructor
 {\Env{}(Θ₁, ρ, Θ₂, Γ)
}{\Env{}(Θ₁ ∙ \texttt{stale}_σ, ρ \texttt{∙v}, Θ₂ ∙ \texttt{stale}_σ, Γ ∙ \texttt{stale}_σ)
}
\end{mathpar}
\end{definition}

\begin{definition}
A Typing Relation \𝓣{} for a \Nat{}-indexed family $T$ equipped with
a function \texttt{subst} which for all \Nat{}s $k, l$, given an
element $T_k$ and an $\Env{}(k,l)$ returns an element $T_l$ is said to
be stable under substitution if for all \Nat{}s $k$ and $l$, γ a $\Context{}_k$,
Γ and Δ two $\Usages{γ}$, $t$ an element of $T_k$, σ a Type, ρ an $\Env{}(k,l)$,
$θ$ a $\Context{}_l$ and $Θ₁$ and $Θ₃$ two $\Usages{θ}$ such that
\𝓣{}$(Γ, t, σ, Δ)$ and $\Env{}(Θ₁, ρ, Θ₃, Γ)$ holds then there exists a $Θ₂$
of type $\Usages{θ}$ such that \𝓣{}$(Θ₁, \texttt{subst}(t, ρ), σ, Θ₂)$ and
$\Env{}(Θ₂, ρ, Θ₃, Δ)$.
\end{definition}

\begin{theorem}\label{theorem:substituting}
The Typing Relations for \Inferable{} and \Checkable{} are stable under substitution.
\end{theorem}
\begin{proof}
The proof by mutual structural induction on the typing derivations relies
heavily on the fact that these Typing Relations enjoy the framing property
in order to readjust the usages annotations.
\end{proof}


%%%%%%%%%%%%%%%%%%
%% FUNCTONALITY %%
%%%%%%%%%%%%%%%%%%

\section{Functionality}

One thing we did not mention before because it was seldom used in the
proofs is the fact that all of these typing relations are functional
when seen as various (binary or ternary) relations. Which means that if
two typing derivations exist for some fixed arguments (seen as inputs)
then the other arguments (seen as outputs) are equal to each other.

\begin{definition}We say that a relation $R$ of type
$Π(\mathit{ri} : \mathit{RI}).Π(\mathit{ii} : \mathit{II}).O(ri) → \Set{}$
is functional if for all relevant inputs $\mathit{ri}$, all pairs of
irrelevant inputs $\mathit{ii₁}$ and $\mathit{ii₂}$ and for all pairs
of outputs $o₁$ and $o₂$, if both $R(\mathit{ri}, \mathit{ii₁}, o₁)$
and $R(\mathit{ri}, \mathit{ii₂}, o₂)$ hold then $o₁ ≡ o₂$.
\end{definition}

\begin{lemma}The Typing Relations for \Var{} and \Inferable{} are functional
when seen as relations with relevant inputs the context and the scrutinee
(either a \Var{} or an \Inferable{}), irrelevant inputs their $\Usages{}$
annotation and outputs the inferred \Type{}s.
\end{lemma}

\begin{lemma}The Typing Relations for \Var{}, \Inferable{}, \Checkable{}
and \Env{} are functional when seen as relations with relevant inputs all
of their arguments except for one of the $\Usages{}$ annotation or the other.
This means that given a $\Usages{}$ annotation (whether the input one or the
output one) and the rest of the arguments, the other $\Usages{}$ annotation
is uniquely determined.
\end{lemma}

%%%%%%%%%%%%%%%%%%
%% TYPECHECKING %%
%%%%%%%%%%%%%%%%%%

\section{Typechecking}

\begin{theorem}[Decidability of Typechecking]
\label{theorem:typechecking}
Type-inference for \Inferable{} and Type-checking for \Checkable{} are
decidable. In other words, given a \Nat{} $k$, γ a $\Context{}_k$ and
Γ a $\Usages{γ}$,
\begin{enumerate}
  \item for all $\Inferable{}_k$ $t$, it is decidable whether there is
        a \Type{} $σ$ and Δ a $\Usages{γ}$ such that $Γ ⊢ t ∈ σ ⊠ Δ$
  \item for all \Type{} σ and $\Checkable{}_k$, it is decidable whether
        there is Δ a $\Usages{γ}$ such that $Γ ⊢ σ ∋ t ⊠ Δ$.
\end{enumerate}
\end{theorem}
\begin{proof}
The proof proceeds by mutual induction on the raw terms, using inversion
lemmas to dismiss the impossible cases, using auxiliary lemmas showing
that typechecking of \Var{}s and \Pattern{}s also is decidable and relies
heavily on the functionality of the various relations involved.
\end{proof}

One of the benefits of having a formal proof of a theorem in Agda is that
the theorem actually has computational content and may be run:

\begin{example}We can for instance check that the search procedure
succeeds in finding the \texttt{swapTyped} derivation we had written down
as Example~\ref{example:swapTyped}. Because σ and τ are abstract in the
following snippet, the equality test checking that σ is equal to itself
and so is τ does not reduce and we need to rewrite by the proof
\texttt{eq-diag} that the equality test always succeeds in this kind of
situation:
\begin{lstlisting}
  swapChecked : '∀' 'σ' 'τ' '→' check [] (('σ ⊗ τ') '⊸' ('τ ⊗ σ')) swap
                      '≡' yes ([] , swapTyped)
  swapChecked 'σ' 'τ' rewrite eq-diag 'τ' | eq-diag 'σ' = refl
\end{lstlisting}
\end{example}


%%%%%%%%%%%%%%%%%%%%%%%%
%% EQUIVALENCE TO ILL %%
%%%%%%%%%%%%%%%%%%%%%%%%

\section{Equivalence to ILL}

We have now demonstrated that the $\Usages{}$-based formulation of
linear logic as a type system is amenable to mechanisation without
putting an unreasonable burden on the user. Indeed, the system's
important properties can all be dealt with by structural induction
and the user still retains the ability to write simple $λ$-terms
which are not cluttered with structural rules.

However this presentation departs quite a lot from more traditional
formulations of intuitionistic linear logic. This naturally raises
the question of its correction. In this section we recall a typical
presentation of Intuitionistic Linear Logic using a Sequent Calculus.

\subsection{A Sequent Calculus for Intuitionistic Linear Logic}

\subsection{Soundness}


\subsection{Completeness}

Completeness is a trickier thing to prove: given a derivation in
the traditional sequent calculus, we need to build a corresponding
term and its typing derivation.

The hardest rule to handle is the \texttt{mix} rule which reorganises
the context. Indeed, its use means that both the term and the
corresponding derivation built by the induction hypothesis need to
be altered in a radical manner.




%%%%%%%%%%%%%%%%%%
%% RELATED WORK %%
%%%%%%%%%%%%%%%%%%

\section{Related Work}

We have already mentioned McBride's work~\cite{mcbride2016got}
on (as a first approximation: the setup is actually more general)
a type theory with a \emph{dependent linear} function space as a
very important source of inspiration. In that context it is indeed
crucial to retain the ability to talk about a resource even if it
has already been consumed. E.g. a function taking a boolean and
deciding whether it is equal to \texttt{tt} or \texttt{ff} will
have a type mentioning the function's argument twice. But in a
lawful manner: $(x : \DefinedType{Bool}) ⊸ (x ≡ \texttt{tt}) ∨ (x ≡ \texttt{ff})$.
This leads to the need for a context \emph{shared} across all
subterms and consumption annotations ensuring that the linear
resources are never used more than once.

We can find a very concrete motivation for a predicate similar to
our $\Usage{}$ in Robbert Krebbers' thesis~\cite{krebbers2015thesis}.
In section 2.5.9, he describes one source of undefined behaviours
in the C standard: the execution order of expressions is unspecified
thus leaving the implementers with absolute freedom to pick any order
they like if that yields better performances. To make their life
simpler, the standard specifies that no object should be modified
more than once during the execution of an expression. In order to
enforce this invariant, Krebbers' memory model is enriched with extra
information:
\begin{quote}
  [E]ach bit in memory carries a permission that is set to a special
  locked permission when a store has been performed. The memory
  model prohibits any access (read or store) to objects with locked
  permissions. At the next sequence point, the permissions of locked
  objects are changed back into their original permission, making
  future accesses possible again.
\end{quote}

\section{Conclusion}

We have shown that taking seriously the view of linear logic as a
logic of resource consumption leads, in type theory, to a well-behaved
presentation of the corresponding type system for the lambda-calculus.
The framing property claims that the state of irrelevant resources does
not matter, stability under weakening shows that one may even add extra
irrelevant assumptions to the context and they will be ignored whilst
stability under substitution guarantees subject reduction with respect
to the usual small step semantics of the lambda calculus. Finally, the
decidability of type checking makes it possible to envision a user-facing
language based on raw terms and top-level type annotations where the
machine does the heavy lifting of checking that all the invariants are
met whilst producing a certified-correct witness of typability.

Avenues for future work include a treatment of an \emph{affine} logic
where the type of substitution will have to be be different because
of the ability to throw away resources without using them. Our long
term goal is to have a formal specification of a calculus for Probabilistic
and Bayesian Reasoning similar to the affine one described by Adams
and Jacobs~\cite{Adams2015Type}.
Another interesting question is whether these resource annotations
can be used to develop a fully formalised proof search procedure for
intuitionistic linear logic. The author and McBride have made an effort
in such a direction~\cite{Allais2015Proof} by designing a sound and
complete search procedure for a fragment of intuitionistic linear logic
with type constructors tensor and with. Its extension to lolipop is
currently an open question.

%%
%% Bibliography
%%

%% Either use bibtex (recommended),

\bibliography{main}

%% .. or use the thebibliography environment explicitely

\appendix
\appendix{Typing derivation for \texttt{swap}}
\label{typing:swap}
\begin{mathpar}
\inferrule
 {\inferrule
   {\inferrule
     {\inferrule
       {
      }{\texttt{fresh}_{σ ⊗ τ} ⊢ \texttt{zero} ∈ σ ⊗ τ ⊠ \texttt{stale}_{σ ⊗ τ}
      }
     }{\texttt{fresh}_{σ ⊗ τ} ⊢ \texttt{var}(\texttt{zero}) ∈ σ ⊗ τ ⊠ \texttt{stale}_{σ ⊗ τ}
   }
   \inferrule
    {\inferrule{ }{σ ∋ \texttt{v} \leadsto{} [] ∙ \texttt{fresh}_σ}
      \and \inferrule{ }{τ ∋ \texttt{v} \leadsto{} [] ∙ \texttt{fresh}_τ}
   }{σ ⊗ τ ∋ (\texttt{v, v}) \leadsto{} [] ∙ \texttt{fresh}_τ ∙ \texttt{fresh}_σ
   } \and Π
  }{[] ∙ \texttt{fresh}_{σ ⊗ τ} ⊢ τ ⊗ σ ∋
    {\begin{array}{l}
      \texttt{let}~(\texttt{v, v})~∷=~\texttt{var}~\texttt{zero}~\texttt{in} \\
      \texttt{prd}(\texttt{neu}(\texttt{var}(1)),~\texttt{neu}(\texttt{var}(0)))
    \end{array}}
    ⊠ [] ∙ \texttt{stale}_{σ ⊗ τ}$$
  }
}{[] ⊢ (σ ⊗ τ) ⊸ (τ ⊗ σ) ∋ \texttt{swap} ⊠ []
}

\and
Π = \inferrule
    {\inferrule
      {\inferrule
        {\inferrule
        {\inferrule
        {
        }{[] ∙ \texttt{fresh}_τ ⊢ 0 ∈ τ ⊠ [] ∙ \texttt{stale}_τ
       } 
        }{[] ∙ \texttt{fresh}_τ ∙ \texttt{fresh}_σ ⊢ 1 ∈ τ
          ⊠ [] ∙ \texttt{stale}_τ ∙ \texttt{fresh}_σ
       } 
        }{[] ∙ \texttt{fresh}_τ ∙ \texttt{fresh}_σ ⊢ \texttt{var}(1) ∈ τ
          ⊠ [] ∙ \texttt{stale}_τ ∙ \texttt{fresh}_σ
       } 
     }{[] ∙ \texttt{fresh}_τ ∙ \texttt{fresh}_σ ⊢ τ ∋ \texttt{neu}(\texttt{var}(1))
       ⊠ [] ∙ \texttt{stale}_τ ∙ \texttt{fresh}_σ
     } \and \inferrule
      {\inferrule
      {
     }{[] ∙ \texttt{stale}_τ ∙ \texttt{fresh}_σ ⊢ 0 ∈ σ
       ⊠ [] ∙ \texttt{stale}_τ ∙ \texttt{stale}_σ
     }
     }{[] ∙ \texttt{stale}_τ ∙ \texttt{fresh}_σ ⊢ \texttt{var}(0) ∈ σ
       ⊠ [] ∙ \texttt{stale}_τ ∙ \texttt{stale}_σ
     }
    }{[] ∙ \texttt{fresh}_τ ∙ \texttt{fresh}_σ ⊢ τ ⊗ σ ∋
      \texttt{prd}(\texttt{neu}(\texttt{var}(1)),~\texttt{neu}(\texttt{var}(0))
      ⊠ [] ∙ \texttt{stale}_τ ∙ \texttt{stale}_σ
   }
\end{mathpar}

\begin{mathpar}
\inferrule{\inferrule{\inferrule{\inferrule{ }{[] ∙ \fresh{α} ⊢ 0 ∈ \fresh{α} ⊠ [] ∙ \stale{α}}}{[] ∙ \fresh{α} ⊢ \var{0} ∈ α ⊠ [] ∙ \stale{α}}}{[] ∙ \fresh{α} ⊢ α ∋ \neu{\var{0}} ⊠ [] ∙ \stale{α}}}{[] ⊢ α ⊸ α ∋ \lam{\neu{\var{0}}} ⊠ []}
\end{mathpar}


\end{document}
