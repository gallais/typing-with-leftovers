\documentclass{beamer}
\usetheme{ru}
\usepackage{xcolor}
\usepackage{amssymb}
\usepackage{mathpartir}

\newcommand{\inval}[1]{\colorbox{blue!40}{$#1$}}
\newcommand{\outval}[1]{\colorbox{orange!50}{$#1$}}
\newcommand{\fresh}[1]{f_{#1}}
\newcommand{\stale}[1]{s_{#1}}

\title{Typing with Leftovers}
\subtitle{A Mechanisation of Intuitionistic Linear Logic}
\author{Guillaume Allais}
\date{May 29th, 2017}
\institute[TYPES'17]{
  TYPES 2017 \\
  Budapest, Hungary}

\begin{document}

\maketitle

\begin{frame}{Introduction rule for tensor: Contexts as Multisets}

\begin{mathpar}
\inferrule
 {Γ ⊢ σ \and Δ ⊢ τ
}{Γ, Δ ⊢ σ ⊗ τ
}{⊗_i}
\end{mathpar}

\begin{itemize}
 \item<2-> Multisets are an extensional notion
 \item<3-> Going bottom-up (typechecking), one needs to guess a split
\end{itemize}
\end{frame}

\begin{frame}{Introduction rule for tensor: Contexts as Lists}
\begin{itemize}
\item Either explicit commutation rules
\begin{mathpar}
\inferrule
 {Γ, σ, τ, Δ ⊢ ν
}{Γ, τ, σ, Δ ⊢ ν
}{swap}
\end{mathpar}

\item Or introduction rules with reordering ``on the fly'':
\begin{mathpar}
\inferrule
 {Γ ⊢ σ \and Δ ⊢ τ \and Γ, Δ ≈ Θ
}{Θ ⊢ σ ⊗ τ
}{⊗_i}
\end{mathpar}
\end{itemize}

\begin{itemize}
 \item<2-> Non structural
 \item<2-> The user has to provide the split / permutation by hand
\end{itemize}
\end{frame}

\begin{frame}{Introduction rule for tensor: Contexts as Available Resources}
\begin{mathpar}
\inferrule
 {\inval{Γ} ⊢ \inval{σ} ⊠ \outval{Δ} \and \inval{Δ} ⊢ \inval{τ} ⊠ \outval{Θ}
}{\inval{Γ} ⊢ \inval{σ ⊗ τ} ⊠ \outval{Θ}
}{⊗_i}
\end{mathpar}

\begin{itemize}
 \item \inval{Inputs} vs. \outval{Outputs}
 \item<2-> Nothing to guess: syntax directed
 \item<3-> However the scope keeps changing!
\end{itemize}
\end{frame}

\begin{frame}{Variant: Contexts as Resource Annotations}
\begin{mathpar}
\inferrule
 {n : ℕ \and
  γ : Context_n \and
  Γ, Δ : Usage_γ \and
  σ : Type
}{
\inval{Γ} ⊢ \inval{σ} ⊠ \outval{Δ} : Set
}
\end{mathpar}

\begin{columns}
\begin{column}{0.5\textwidth}
\begin{mathpar}
\inferrule
 {
}{
\inval{Γ, \fresh{σ}} ⊢ \inval{σ} ⊠ \outval{Γ, \stale{σ}}
}{ax}
\end{mathpar}
\end{column}
\begin{column}{0.5\textwidth}
\begin{mathpar}
\inferrule
 {\inval{Γ} ⊢ \inval{σ} ⊠ \outval{Δ}
}{
\inval{Γ, s} ⊢ \inval{σ} ⊠ \outval{Δ, s}
}{wk}
\end{mathpar}
\end{column}
\end{columns}

\begin{itemize}
 \item<2-> Context shared across the whole derivation
 \item<3-> The Axiom rule corresponds to consumption
 \item<4-> A notion of weakening: some resources don't matter (yet)
\end{itemize}
\end{frame}

\begin{frame}{With Well Scoped Terms, Bidirectionally}
\begin{columns}
\begin{column}{0.5\textwidth}
\begin{mathpar}
\inferrule
 {n : ℕ \and
  γ : Context_n \\
  t : Check_n \\
  Γ, Δ : Usage_γ \\
  σ : Type
}{
\inval{Γ} ⊢ \inval{σ} ∋ \inval{t} ⊠ \outval{Δ} : Set
}
\end{mathpar}
\end{column}
\begin{column}{0.5\textwidth}
\begin{mathpar}
\inferrule
 {n : ℕ \and
  γ : Context_n \\
  t : Infer_n \\
  Γ, Δ : Usage_γ \\
  σ : Type
}{
\inval{Γ} ⊢ \inval{t} ∈ \outval{σ} ⊠ \outval{Δ} : Set
}
\end{mathpar}
\end{column}
\end{columns}

\begin{itemize}
 \item<2-> Fully syntax-directed (same constructors for $λ$C \& TwL)
 \item<3-> Fully compatible with operations on $λ$C
\end{itemize}
\end{frame}

\begin{frame}{Theorems}
\begin{tabular}{ll}
 Framing       & Unused resources can be in any state \\
 Weakening     & One can add resources, they will be ignored \\
 Substitution  & Parallel linear substitutions yield valid terms \\
 Functionality & The relation is deterministic (backward \& forward) \\
 Typechecking  & Type checking / inference is decidable \\
 Soudness      & Derivations in TwL give rise to sequents in ILL \\
 Completeness  & Sequents in ILL give rise to derivations in TwL \\
\end{tabular}
\end{frame}

\begin{frame}[fragile]{Thanks for Your Attention}
All of this work has been formalised in Agda.
It is available at:

\url{https://github.com/gallais/typing-with-leftovers}
\end{frame}

\end{document}
